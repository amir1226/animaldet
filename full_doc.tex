\documentclass[11pt,a4paper]{article}
\usepackage[utf8]{inputenc}
\usepackage{graphicx}
\usepackage{hyperref}
\usepackage{listings}
\usepackage{float}

\title{Animaldet Documentation}
\date{}

\begin{document}

\maketitle

\section{About the team}

The dream team is integrated by:

\begin{itemize}
    \item Amir Sadour
    \item Camilo Rodriguez
    \item Claudia Agudelo
    \item Luis Manrique
\end{itemize}

\section{Live demo}

You can see a demo in this \href{http://animaldet-alb-510958915.us-east-1.elb.amazonaws.com/}{link}

\section{Repository}

Your code could be found on \href{https://github.com/amir1226/animaldet}{github}

\section{General animaldet installation guide}

Animaldet was built as a containerized application which can run locally or at any cloud which support containers

\section{Local deployment}

First of all with the repo cloned you must execute
\begin{lstlisting}[language=bash]
git lfs pull
\end{lstlisting}

To get models and other useful information to deploy your model.

You should have \texttt{docker} installed.

After that you can execute

\begin{lstlisting}[language=bash]
make build
docker run -p 8000:8000 animaldet:latest
\end{lstlisting}

Then when opening in your browser \texttt{localhost:8000} you must see the application running.

\begin{figure}[H]
\centering
\includegraphics[width=0.8\textwidth]{static/img/main_view.png}
\caption{Main view}
\end{figure}

For user guide about how to use Animaldet you can check the \href{USER_GUIDE.md}{next link}

\section{Cloud deployment}

Animaldet included deployment platform is \texttt{AWS ECS}, with the next architecture:

\begin{figure}[H]
\centering
\includegraphics[width=0.8\textwidth]{static/img/architecture.png}
\caption{Architecture}
\end{figure}

We included all the required infrastructure as a code at \texttt{infra/aws}, there we use \texttt{terraform}.

In order to deploy the application you must configure your aws credentials, more information at \href{https://docs.aws.amazon.com/cli/v1/userguide/cli-configure-files.html}{aws cli configuration documentation}

Commands to deploy the application

\begin{lstlisting}[language=bash]
make deploy
\end{lstlisting}

It will execute \texttt{docker build}, then \texttt{terraform apply} which is going to:

\begin{enumerate}
    \item Build docker image locally
    \item Login you to ECS registry
    \item Push the image to ECS
    \item Create clusters, task definitions, policies and load balancer to get the application working.
\end{enumerate}

\section{Animaldet: how to use our animal detection tool}

First of all, the main screen:

\begin{figure}[H]
\centering
\includegraphics[width=0.8\textwidth]{static/img/main_view.png}
\caption{Main screen}
\end{figure}

You may chose your own images or select one of the sample ones.

If you decide to pick one of the samples, you are going to see the image annotations and a button to predict executions:

\begin{figure}[H]
\centering
\includegraphics[width=0.8\textwidth]{static/img/guide/sample_chosen.png}
\caption{Sample chosen}
\end{figure}

You can also pick between \texttt{small} and \texttt{nano} model, and change the model confidence.

\subsection{Model size}

Nano model uses 30\% less ram than Small model, but as it uses a lower resolution it must perform more inferences over the whole image, inference times does not differ much between those models.

\subsection{Confidence}

Each model has its own suggested confidence, confidence is a model output where it told us how confident the model is about a found bounding box with its respective class, your recommended confidence is the one which maximizes performance but you may experiment with different values

\subsection{Performing inference}

Once the image and parameters are chosen, you can click then \texttt{Detect Animals} button, and a loading screen must start

\begin{figure}[H]
\centering
\includegraphics[width=0.8\textwidth]{static/img/guide/loading.png}
\caption{Loading screen}
\end{figure}

Once finished it must show the image detections:

\begin{figure}[H]
\centering
\includegraphics[width=0.8\textwidth]{static/img/guide/detections.png}
\caption{Detections}
\end{figure}

\section{Low confidence: over detection}

When you pick a lower confidence you may expect over prediction as shown in the next capture:

\begin{figure}[H]
\centering
\includegraphics[width=0.8\textwidth]{static/img/guide/overdetection.png}
\caption{Over detection}
\end{figure}

It may work for experimentation purposes, but please be careful.

\section{Experiment comparison: lets see how the model performed}

We have added an experiments view where you can see detections over the whole test dataset compared with the ground truth:

\begin{figure}[H]
\centering
\includegraphics[width=0.8\textwidth]{static/img/guide/experiments.png}
\caption{Experiments view}
\end{figure}

There you can pick any image from the test set and check the difference between the ground truth and your predictions.

\end{document}
